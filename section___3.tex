\section{Κατάρρευση περιστρεφόμενου και μαγνητισμένου μοριακού πυρήνα}
Στα προηγούμενα όπου θεωρήσαμε ένα αρχικό σφαιρικό, μη-περιστρεφόμενο, μη-μαγνητισμένο πυρήνα, ασχολούμασταν με την "διαμάχη" της βαρύτητας με τη θερμική πίεση του αερίου. 


Στη περίπτωση όπου το νέφος έχει μια αρχική γωνιακή ταχύτητα, βρίσκεται μέσα σε μαγνητικό πεδίο και έχει μάζα μεγαλύτερη από τη κρίσιμη (βλέπε και παράγραφο~\ref{par:VirialMass}) θα καταρρεύσει σε γενικές γραμμές όπως αναλύσαμε στη κλασική περίπτωση.


Η μεγαλύτερη διαφορά με τη κλασική προσέγγιση εμφανίζεται κυρίως στη φάση της πρόσπτωσης της ύλης στο πρωτοαστέρα όπου η περιστροφή και το μαγνητικό πεδίο παίζει σημαντικό ρόλο.


\subsection{Η περίπτωση αργά περιστρεφόμενου πυρήνα}\todo{γιατί και πόσο αργά?}
Θεωρούμε ότι αρχικά το νέφος είναι αρχικά σφαιρικά συμμετρικό με μικρή γωνιακή ταχύτητα και ότι η κατάρρευση έχει δημιουργήσει ήδη τον κεντρικό πρωτοαστέρα.
Το υλικό εκτελεί ελεύθερη πτώση ξεκινώντας απο τη θέση $(r_0,\theta _0)$, όπου $\theta$ η γωνία από τον άξονα περιστροφής, προς το κεντρικό πυρήνα διατηρώντας την ειδική στροφορμή του $j=\Omega r_0 ^2 \sin \theta_0$, και με σταθερή επιτάχυνση $\dot{M} \simeq \frac{c_s ^3}{G}$.   


Αν η κατάρρευση διατηρηθεί συμμετρική ώς προς τις γωνίες $\phi$ \todo{πως το λεμε αυτό?} και ως προς τη ισημερινή επιφάνεια $\theta = \pi/2$ είναι προφανές ότι το υλικό "από πάνω" θα συγκρουστεί με το "από κάτω" πάνω στο στο ισημερινό επίπεδο, και μάλιστα αποδεικνύεται ότι για κάθε αρχικό σημείο $(r_0,\theta _0)$ το σημείο της σύγκρουσης αντιστοιχεί σε ένα σημείο $(r_{ct},\pi/2)$, όπου:
\begin{equation}
r_{ct}=\frac{j^2}{GM}=\frac{\Omega^2 r_0 ^4 \sin ^2 \theta_0}{GM}
\end{equation}
Το αποτέλεσμα θα είναι η δημιουργία μια κατανομής πυκνότητας:
\begin{equation}
\rho (r,\theta) =\frac{\dot{M}}{4 \pi \sqrt{G M r^3}}\left(1+\frac{\cos \theta}{\cos \theta _0}\right)^{-1/2} \left( \frac{\cos \theta}{\cos \theta _0} + \frac{2 R_c \cos^2 \theta _0}{r} \right) ^{-1} 
\end{equation}


Μέσω αυτής της διαδικασίας έχουμε τη δημιουργία ενός δίσκου προσαύξησης όπου η ύλη ακολουθεί κεπλεριανές τροχιές γύρω από τον αστέρα.


Άρα βλέπουμε ότι τα σωματίδια με μικρές αρχικές γωνίες $\theta _0 \to 0$, δηλαδή με μικρή στροφορμή, θα συγκρουστούν πάνω στην επιφάνεια του πρωτοαστέρα, ενώ τα σωματίδια που βρίσκονται από την αρχή στο ισημερινό επίπεδο θα είναι αυτά που θα δημιουργήσουν τις τελευταίες τροχίες του. 


Έτσι μπορούμε να υπολογίσουμε την ακτίνα ολόκληρου του δίσκου:
\begin{equation}
R_c=\frac{\Omega^2 r_0 ^4}{GM} = m_0 \frac{\Omega^2 (c_s t)^4}{G \dot{M} t} =m_0 \Omega ^2 c_s t^3
\end{equation}
όπου $m_0$ μια σταθερά, η οποία βρίσκεται από την αναλυτική λύση $m_0=0.058$. 


Για τυπικές τιμές ενός πρωτοαστέρα, ($M=1$, $M_{\odot}$, $\dot{M}=10^{-5} \ M_{\odot} \, yr^{-1}$, $c_s =0.35\ km\, s^{-1}$, $r_0 = 1.5 \e{15}\ cm$) βρίσκουμε ότι $R_c \simeq 44\ A.U.$.  


\subsection{Περίπτωση κατάρρευσης μαγνητισμένου νέφους}
Στη παράγραφο \ref{par:frozenmagneticfield} αναφέραμε τη "κοινή" συμπεριφορά μαγνητικού πεδίου και ιονισμένης ύλης. Στη συνέχεια θα αναφερθούμε περιγραφικά στο πως αυτή η συμπεριφορά επιδρά στη βαρυτική κατάρρευση του πυκνού μοριακού πυρήνα.


Το υπό κατάρρευση μοριακό νέφος αποτελείται σε πολύ μικρό ποσοστό από ιονισμένη ύλη. Άρα δεν μπορούμε να ισχυριστούμε τη προσέγγιση του παγωμένου μαγνητικού πεδίου. 


Σε αυτή τη περίπτωση έχουμε δύο διαφορετικές κινήσεις μέσα στο νέφος: το ιονισμένο αέριο κινείται κατά μήκος των μαγνητικών γραμμών, ενώ το ουδέτερο κινείται λόγω της βαρύτητας και της θερμικής πίεσης, ανεξάρτητα του μαγνητικού πεδίου.
Όμως οι δύο αυτοί πληθυσμοί σωματιδίων εφόσον αλληλεπιδρούν μεταξύ τους μέσω συγκρούσεων, επηρεάζουν εν τέλει και τις δύο αυτές κινήσεις. Το φαινόμενο αυτό ονομάζεται \textbf{διπολική διάχυση}.


Έστω ότι ο πυκνός μοριακό πυρήνας βρίσκεται εντός ενός ομογενούς αρχικού μαγνητικού πεδίου $B_0$. Καθώς ξεκινάει η κατάρρευση τα ιονισμένα και τα ουδέτερα σωματίδια είναι ελεύθερα να κινηθούν κατά μήκος των μαγνητικών δυναμικών γραμμών. 


Στη διεύθυνση κάθετα στο μαγνητικό πεδίο όμως, τα ουδέτερα σωματίδια συγκρούονται με τα ιονισμένα, με αποτέλεσμα η κατάρρευση σε αυτή τη διεύθυνση να επιβραδύνεται και τα ιόντα να παρασύρονται από τα ουδέτερα. Όμως καθώς το μαγνητικό πεδίο είναι παγωμένο μέσα στην ιονισμένη ύλη, οι δυναμικές του γραμμές συγκλίνουν και αυτές προς το κέντρο κατάρρευσης, αυξάνοντας τοπικά το μαγνητικό πεδίο, ενώ η ίδια η καμπύλωση τους ασκεί δύναμη αντίθετη στη κατάρρευση λόγω της μαγνητική τάσης. 


Εν τέλει, όπως και με τη περίπτωση του περιστρεφόμενου πυρήνα, το μαγνητικό πεδίο δημιουργεί ένα δίσκο γύρω από το πρωτοαστέρα, που όμως δεν έχει της δυναμικές ιδιότητες του δίσκου προσαύξησης.
