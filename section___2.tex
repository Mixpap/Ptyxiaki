\section{Κατάρρευση του μοριακού Πυρήνα}
Η δημιουργία των νέων αστέρων σύμφωνα με όλες τις ενδείξεις τροφοδοτείτε από τη βαρυτική κατάρρευση των πυκνότερων περιοχών των μοριακών νεφών, των μοριακών πυρήνων. Παρακάτω θα αναφερθούμε, χωρίς να επεκταθούμε, στις κυρίαρχες διαδικασίες κατάρρευσης αυτών των πυρήνων.


\subsection{Αρχικές συνθήκες}
Σαν αρχικές συνθήκες της κατάρρευσης του πυκνού μοριακού πυρήνα θα χρησιμοποιήσουμε τα τυπικά φυσικά χαρακτηριστικά όπως έχουν ανιχνευθεί από παρατηρήσεις αλλά και στα κάποιες θεωρητικές προσεγγίσεις με βάση αυτά.


\begin{tabular}
	Μάζα: & $1$ \sm \\
	Ακτίνα: & $0.1 \ pc$ \\
	Θερμοκρασία: & $10 \ K$ \\
	Πυκνότητα: & $10^{-19} \ g \, cm^{-3}$ \\
	Ποσοστό Ιονισμού: & $10^{-7}$ 
\end{tabular}


\subsubsection{Σφαίρα Bonnor-Ebert}
Η σφαίρα Bonnor-Ebert είναι η θεωρητική κατασκευή μιας ισόθερμης σφαίρας όπου η βαρύτητα εξισορροπείται από την εσωτερική πίεση. Δηλαδή ισχύουν οι εξισώσεις:
\begin{align}
\frac{Gm}{r^2} &+\frac{1}{\rho}\frac{dP}{dr}=0 \text{ Εξίσωση Κίνησης}\\
\frac{dm}{dr} &= 4 \pi r^2 \rho \text{ Εξίσωση διατήρησης της Μάζας}\\
P &= c_s ^2 \rho \text{ Καταστατική Εξίσωση}
\end{align}


Συνδυάζοντας και τις τρείς έχουμε:
\begin{equation}
\frac{1}{r^2}\frac{d}{dr} \left( r^2 c_s ^2 \frac{d \ln \rho}{dr}\right)  = -4 \pi G \rho
\end{equation}


Η λύση της οποίας μας δίνει τη πυκνότητα συναρτήση της ακτίνας μέσα στο μοριακό πυρήνα:
\begin{equation}
\label{eq:B-E_density}
\rho (r) =\frac{c_s ^2}{2 \pi G} \frac{1}{r^2}
\end{equation}


\subsection{Μαγνητικά πεδία και ύλη}
\label{par:frozenmagneticfield} 

Όπως έχουμε αναφέρει και στα προηγούμενα τα μαγνητικά πεδία φαίνονται να παίζουν σημαντικό ρόλο στη διαδικασία της βαρυτικής κατάρρευσης αλλά κυρίως στην αλληλεπίδραση του πρωτοαστέρα με το άμεσο περιβάλλον του.


Η προσέγγιση που έχουμε για την αλληλεπίδραση της ιονισμένης ύλης με τα μαγνητικά πεδία, είναι αυτή της μαγνητοϋδροδυναμικής, δηλαδή τη σύνδεση των υδροδυναμικών εξισώσεων διατήρησης (μάζα, ορμή, ενέργεια) με τις εξισώσεις του Maxwell.  


Η πολυπλοκότητα των εξισώσεων αυτών επιτρέπει την ακριβή λύση μόνο ειδικών υπεραπλουστευμένων περιπτώσεων, γι αυτό και χρησιμοποιούνται κυρίως αριθμητικοί κώδικες για την επίλυση τους.


Όμως εκτός από τη πολυπλοκότητα, ένα άλλο σοβαρό πρόβλημα που αντιμετωπίζουμε είναι η δυσκολία στο να μετρήσουμε και να χαρτογραφήσουμε το μαγνητικό πεδίο στα μοριακά νέφη και ειδικότερα στους πυκνούς πυρήνες που μελετάμε σε αυτό το κεφάλαιο. \todo{χαρακτηριστική τιμή} 


\subsubsection{Εξίσωση Επαγωγής}
Το μαγνητικό πεδίο προσφέρει στα μοριακά νέφη ακόμα μια δύναμη υποστήριξης, μαζί με την θερμική πίεση και τη περιστροφή, απέναντι στη βαρύτητα. Η φυσική βάση που επιτρέπει στα μαγνητικά πεδία να πέρνουν ενεργό μέρος σε αυτή τη διαδικασία είναι το φαινόμενο του "παγώματος" του μαγνητικού πεδίου μέσα στην ύλη.


Το φαινόμενο αυτό "συνδέει" την ύλη με το μαγνητικό πεδίο, έτσι καθώς η πρώτη συμπιέζεται λόγω βαρύτητας συμπιέζει μαζί της και τις δυναμικές γραμμές του πεδίου με αποτέλεσμα αυτό τοπικά να αυξάνεται.


Για να δούμε πως χτίζεται το φαινόμενο του "παγωμένου" μαγνητικού πεδίου πρέπει να κοιτάξουμε την εξίσωση επαγωγής:
\begin{equation}
\pt{\bb} = \nn \times (\vv \times \bb) -\nn \times \left( \frac{c^2}{4 \pi \sigma} \nn \times \bb \right) 
\end{equation}
όπου  $\bb$ το μαγνητικό πεδίο και $\sigma$ η είδική ηλεκτρική αγωγιμότητα.
Η εξίσωση της επαγωγής μας δίνει τη χρονική μεταβολή του πεδίου συναρτήσει ενός όρου μεταφοράς, δηλαδή τη μεταβολή της ροής του ιονισμένου υλικού, και ενός όρου διάχυσης (ωμική διάχυση).


Για ένα μοριακό νέφος παρά το χαμηλό ποσοστό ιονισμένης ύλης (σε σχέση με την ουδέτερη) αποδεικνύεται ότι ο όρος διάχυσης είναι αμελητέος \footnote{Λόγω των κρούσεων του ιονισμένου και του ουδέτερου υλικού, εμφανίζεται ένας άλλος όρος διάχυσης, η διπολική διάχυση, για την οποία θα μιλήσουμε στη συνέχεια.} αρά η χρονική εξέλιξη του μαγνητικού πεδίου καθορίζεται από τη κίνηση του ιονισμένου ρευστού καθιστώντας το "παγωμένο".


\subsection{Ισόθερμη κατάρρευση}
Μόλις ο πυρήνας γίνει βαρυτικά ασταθείς και ξεκινάει να καταρρέει το ενεργειακό πλεόνασμα (που κερδίζεται από τη βαρυτική δυναμική ενέργεια) μετατρέπεται σε θερμότητα η οποία μεταφέρεται από τα μόρια στους κόκκους σκόνης που την απελευθερώνουν μέσω ακτινοβολίας στο Υπέρυθρο με αποτέλεσμα η θερμοκρασία του καταρρέοντος υλικου να παραμένει σταθερή. Αυτή η φάση κατάρρευσης του πυρήνα ονομάζεται \textbf{ισόθερμη φάση} και χαρακτηρίζεται από την ελεύθερη πτώση του υλικού στη κεντρική περιοχή.


\subsubsection{Ελεύθερη Πτώση}
Όσο η κεντρική πυκνότητα παραμένει μικρότερη από $10^{-13} \ g \, cm^{-3}$\footnote{Δηλαδή όσο το οπτικό βάθος είναι $\tau<<1$. Το οπτικό βάθος συνδέεται με τη πυκνότητα από τη σχέση $\tau \simeq k \rho R$ όπου $k$ ο δείκτης αδιαφάνειας} το γύρω υλικό θα καταρρεύσει σε χρονική κλίμακα ελεύθερης πτώσης $t_{ff} \propto (G \rho)^{-1/2} \sim 10^5 \ yr$. 
Ο ρυθμός εισροής μάζας μπορεί να υπολογιστεί μέσω ανάλυσης κλίμακας: 
\begin{equation}
\dot{M} \sim \frac{M}{t_{ff}} \sim \frac{\rho R^3}{(G \rho)^{-1/2}} \sim \frac{\rho \frac{c_s ^3}{(G \rho)^{3/2}}}{(G \rho)^{-1/2}} \sim \frac{c_s ^3}{G} \simeq 10^-6 \  M_{\odot} \, yr^{-1} 
\end{equation}
όπου για την ακτίνα χρησιμοποιήσαμε την ακτίνα $R \sim c_s t_{ff} \sim r_{jeans}$ η οποία διαχωρίζει το προς κατάρρευση υλικό του πυρήνα, με το εξωτερικό.


Από τη σχέση (\ref{eq:B-E_density}) $\rho \sim r^{-2} \rightarrow m \sim r$ εφόσον μας ενδιαφέρει η κατανομή της πυκνότητας στο "εξωτερικό" του πρωτοαστέρα άρα: 
\begin{equation}
\dot{M} \sim \frac{m}{t_{ff}} \sim m \rho^{1/2} \sim const.
\end{equation}
Άρα ο ρυθμός εισροής μάζας είναι σταθερός\footnote{Στη πραγματικότητα ο ρυθμός εισροής δεν είναι σταθερός, γι αυτό και χρησιμοποιείτε η χρονοεξαρτώμενη παραλλαγή $\dot{M}=\frac{c_s ^3}{G} e^{t/\tau}$ όπου $\tau$ μια χρονική κλίμακα όπου ο αστέρας έχει εισέλθει στη κύρια ακολουθία} κατά τη διάρκεια της ισόθερμης κατάρρευσης.


\subsection{Αδιαβατική Κατάρρευση}
Όταν η κεντρική περιοχή ξεπεράσει σε πυκνότητα τα $10^{-13} \ g \, cm^{-3}$ τότε η κατάρρευση σταματάει να είναι ισόθερμη αφού τα εσωτερικά στρώματα του πυρήνα γίνονται οπτικά αδιαφανή μην επιτρέποντας στο πλεόνασμα της ενέργειας να αποδράσει μέσω της ακτινοβολίας. Έτσι η κεντρική θερμοκρασία και η πίεση αυξάνονται.
Στη θερμοκρασία των $1000 \ K$ οι περισσότεροι κόκκοι εξαερώνονται έτσι δεν μπορούν πια να απορροφήσουν τη θερμότητα από τα μόρια που διεγείρονται.


Η καταστατική εξίσωση είναι τώρα αδιαβατική με $\frac{d \log T}{d \log \rho} = (\gamma-1) \simeq 0.4$ για το μοριακό Υδρογόνο με 5 βαθμούς ελευθερίας. Η κεντρική πίεση σε αυτό το σημείο υπερνικάει τη βαρύτητα και η κατάρρευση επιβραδύνεται δημιουργώντας ένα πρωτοαστέρα (δηλαδή ένα κεντρικό πυρήνα με υδροστατική ισορροπία) με μάζα τάξης $10^{-2}$ \sm, θερμοκρασίας $200 \ K$ και πυκνότητας $10^{-10} \ g \, cm^{-3}$.
Η ξαφνική επιβράδυνση της κατάρρευσης δημιουργεί ένα κρουστικό κύμα σε μια ακτίνα $4 \ AU$


Η εισροή μάζας στο πρωτοαστέρα αυξάνει τη πυκνότητα στο πύρηνα του στα $10^{-8} \ g \, cm^{-3}$ σε μια θερμοκρασία $1600 \ K$, όπου το \ce{H2} διασπάται μειώνοντας τον αδιαβατικό δείκτη στη τιμή $\gamma \simeq 1.1$ με αποτέλεσμα την επανεκκίνηση της κατάρρευσης με ταχύτητες αντίστοιχες της ελεύθερης πτώσης. Καθώς ολόκληρο το \ce{H2} διασπάται ο αδιαβατικός δείκτης συγκλίνει κοντά στη τιμή $5/3$ ενός αερίου ουδέτερου \ce{H} και \ce{He}. Η κεντρική θερμοκρασία σε αυτό το στάδιο έχει φτάσει στους $8000 \ K$.


Ο πρωτοαστέρας θα αποκτήσει ξανά υδροστατική ισορροπία όταν η πυκνότητα στο πυρήνα του γίνει $10^{-2} \ g \, cm^{-3}$ και η θερμοκρασία $20000 \ K$. Ο πρωτοαστέρας εξακολουθει να έχει σε αυτό το σημείο μάζα $10^{-2}$ \sm ενώ ένα νέο κρουστικό κύμα δημιουργείται σε απόσταση μερικών $R_{\odot}$. 


\subsection{Φάση Προσαύξησης}
Αν και ο πρωτοαστέρας βρίσκεται πια σε φάση Υδροστατικής ισορροπίας η πλειοψηφία της αρχικής μάζας του μοριακού πυρήνα συνεχίζει να προσαυξάνεται σε αυτόν προσκρούοντας πάνω στο κρουστικό κύμα που περιγράψαμε παραπάνω. 
Η κινητική ενέργεια της ύλης που φτάνει σε αυτό το σημείο μετατρέπεται σχεδόν εξ ολοκλήρου σε ακτινοβολία, δηλαδή $\frac{u^2}{2}=\frac{GM}{R}$, άρα αν πολλαπλασιάσουμε με $\dot{M}$ βρίσκουμε την εισροή ενέργειας ανά δευτερόλεπτο. Αν υποθέσουμε επιπλέον ότι ολόκληρη αυτή η ενέργεια μετατρέπεται σε ακτινοβολία από το κρουστικό κύμα: 
\begin{equation}
L_{acc} \simeq \frac{GM\dot{M}}{R}
\end{equation}


Ταυτόχρονα και ο ίδιος ο πρωτοαστέρας ακτινοβολεί με ρυθμό:
\begin{equation}
L_{star}=4 \pi R^2 \sigma T_{eff} ^4
\end{equation}
σύμφωνα με το νόμο Stefan-Boltzmann.


Για πρωτοαστέρες μικρής και μέσης μάζας η λαμπρότητα λόγω πρόσπτωσης $L_{acc}$ κυριαρχεί έναντι της λαμπρότητας του ίδιου του πρωτοαστέρα.
