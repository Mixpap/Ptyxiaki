\documentclass[a4paper,11pt]{memoir}

\usepackage{fontspec}
\usepackage{float}

\usepackage{polyglossia}
\setmainlanguage{greek}
\setotherlanguage{english}
\setmainfont{Liberation Serif}
%\newfontfamily\greekfont{Liberation Serif}
%\newfontfamily\greekfontsf{Liberation Serif}
%\newfontfamily\greekfonttt{Liberation Serif}

\usepackage{latexsym,graphicx}
%\usepackage{cite}
\usepackage{todonotes}
\usepackage[version=3]{mhchem}
\usepackage[style=nature]{biblatex}
\bibliography{Bibliography.bib}

\newcommand{\n}[1]{\todo[size=\tiny]{#1}}
\newcommand{\dd}{\todo[size=\tiny]{διατύπωση/μεταφραση}}
\newcommand{\bb}{\todo[size=\tiny]{βιβλιογραφία}}
\newcommand{\sm}{$M_{\odot}$}


%memoir page style, ligo allagmeno
%\copypagestyle{myruled}{ruled}
%\makeevenhead{myruled}{\footnotesize\slshape\leftmark}{}{}
%\makeoddhead{myruled}{}{}{\footnotesize\slshape\rightmark}
%\pagestyle{myruled}


\chapterstyle{ger}
%\chapterstyle{lyhne}
%\chapterstyle{madsen}
%\renewcommand*{\chapnumfont}{\normalfont\huge\bfseries}
%\renewcommand*{\chaptitlefont}{\normalfont\huge\bfseries}

\begin{document}


\chapter{Ύλη μεταξύ των αστέρων}
Στον μεσοαστρικό χώρο υπάρχει μια τεράστια ποσότητα ύλης \n{ποσοστό στο γαλαξία?} υπό τη μορφή αερίου και σκόνης. Το υλικό αυτό είναι η πρωτογενής αιτία \dd της δημιουργίας των αστέρων άρα η έρευνα για τη σύνθεση και τα χαρακτηριστικά της είναι απαραίτητη για την βαθύτερη κατανόηση της πρώιμης \dd δημιουργίας των αστέρων.

Σήμερα γνωρίσουμε οτι η ύλη μεταξύ των αστέρων αποτελείται περίπου κατα 99\% απο αέριο και κατα 1\% απο σκόνη με τη συνολική της μάζα στο γαλαξία μας να είναι της τάξης των \sm \n{μάζα αερίου} ενώ η πυκνότητα της κυμαίνεται απο $10^{-4}$ έως $10^{6}$ σωματίδια ανά $cm^3$ \bb.

\section{Φάσεις και χαρακτηριστικά της Μεσοαστρικής Ύλης}
Η Μεσοαστρική Ύλη (ISM) απαντάται σε τρεις φάσεις με διαφορετικά φυσικά και χημικά χαρακτηριστικά 
\footnote{Για τα χημικά χαρακτηριστικά αναφερόμαστε στή σύνθεση των μορίων και στην αναλογία των στοιχείων. Στα φυσικά χαρακτηριστικά αναφερόμαστε στη πυκνότητα και τη θερμοκρασία της Ύλης} 
τη \textbf{ψυχρή} που αποτελείται απο μοριακό και ατομικό αέριο Υδρογόνου και σκόνη, τη \textbf{θερμή} από ατομικό και ιονισμένο άεριο Υδρογόνο και την \textbf{υπέρθερμη} από διεγερμένο αέριο από κρουστικά κύματα εκρήξεων supernova.

\subsection{Ενεργειακή ισορροπία}
Η κινητική θερμοκρασία \footnote{Το ψυχρό μεσοαστρικό αέριο λόγω της γενικά χαμηλής του πυκνότητας δεν βρίσκεται σε θερμοδυναμική ισορροπία. Επομένως όταν μιλάμε για θερμοκρασία αναφερόμαστε στη κινητική του θερμοκρασία.\cite[p. 28]{spitzer_physical_1998}} της Μεσοαστρικής Ύλης κυμαίνεται σε ένα εύρος τιμών 6 τάξεων μεγέθους όπως παρατηρούμε και από τον πίνακα~\ref{tab:ISM}. Για να περιγράψουμε και να μοντελοποιήσουμε την ενεργειακή ισορροπία στη Μεσοαστρική Ύλη άρα να εξηγήσουμε και τις παρατηρούμενες θερμοκρασίες θα πρέπει να υπολογίσουμε τις διαδικασίες θέρμανσης και ψύξης. 
Η κύρια διαδικασία ψύξης είναι η εκπομπή ακτινοβολίας είτε μέσω αυθόρμητης αποδιέγερσης ή αποδιέγερσης λόγω κρούσης. Ενώ για τη θέρμανση έχουμε μια πληθώρα διαδικασιών θέρμανσης οι οποίες μπορούν να ταξινομηθούν σε 3 κατηγορίες:
\begin{itemize}
	\item θέρμανση από πεδία ακτινοβολίας: φωτοηλεκτρική απορρόφηση σε ουδέτερα στοιχεία, φωτοδιάσπαση στα μόρια, φωτοιονισμός.
	\item θέρμανση μέσω συγκρούσεων: από τυρβώδες ροές, κρουστικά κύματα καταλοίπων supernova και κοσμικής ακτινοβολίας.
	\item θερμική ανταλλαγή μεταξύ της σκόνης και νεφών αερίου, αλληλεπίδραση ιονισμένου αερίου με μαγνητικά πεδία, βαρυτική κατάρρευση. 
\end{itemize}

\begin{table}
\caption{Χαρακτηριστικά της μεσοαστρικής ύλης και περιοχές παρατήρησης}
\label{tab:ISM}
\begin{tabular}{p{2.6cm} p{2.5cm}  c  c  p{4.75cm}}
\toprule
Κατηγορία & Κατάσταση Υδρογόνου & $ T \,(K)$ & $ n \,(cm^{-1})$ & Περιοχή Παρατηρήσεων \\ \hline
Μοριακά Νέφη & Μοριακό \ce{H2} & 10-50 & $>10^3$ & Μοριακή εκπομπή - απορρόφηση στο Ράδιο και στο Υπέρυθρο \\
Ψυχρά Νέφη \ce{H I} & Ατομικό \ce{H} & $100$ & $30$ & Γραμμή απορρόφησης $21 \,cm$\\
Θερμό \ce{H I} & Ατομικό \ce{H} & $10^3$ & $0.1$ & Γραμμή εκπομπής $21 \,cm$\\
Θερμό \ce{H IΙ} & Ιονισμένο \ce{H+} & $10^4$ & $10^{-2}$ & Γραμμή Εκπομπής \ce{H\alpha}\\
Περιοχές \ce{H IΙ} & Ιονισμένο \ce{H+}& $10^4$ & $>100$ & Γραμμή Εκπομπής \ce{H\alpha}\\
Υπέρθερμο Ιονισμένο αέριο & Ιονισμένο \ce{H+}& $10^6-10^7$ & $10^{-3}$ & Εκπομπή ακτινοβολίας Χ, Απορρόφηση από ιονισμένα μέταλλα\\
\bottomrule
\end{tabular}
\end{table}

\section{Παρατηρήσεις της Μεσοαστρικής Ύλης}

\section{Το Μεσοαστρικό Αέριο}

Από τις πρώτες κιόλας παρατηρήσεις \footnote{μέσω της απορρόφησης μακρινών λαμπρών πηγών} του μεσοαστρικού αερίου ανακαλύψαμε ότι βρίσκεται κυρίως σε μορφή διακριτών συμπυκνώσεων δηλαδή έχουν δομή νεφών ενώ μπορεί να βρίσκεται σε μορφή ατομική, ιονισμένη και μοριακή.  
Τα συστατικά του αερίου είναι:
\begin{itemize}
	\item Υδρογόνο (\ce{H2}, \ce{H}, \ce{H I}, \ce{H II}, \ce{e-})
	\item Ήλιο (\ce{He I},\ce{He II})
	\item Trace Elements (\ce{C},\ce{O},\ce{Ne},\ce{Mg},\ce{Fe}, κ.α.) \dd 
	\item Μόρια (\ce{CO},\ce{Cs}, κ.α.) 
\end{itemize}
Το κυριότερο σε αναλογία συστατικό του Μεσοαστρικού Αερίου είναι το ουδέτερο ατομικό Υδρογόνο, το οποίο σε θερμοκρασίες $50-200 K$ βρίσκεται σχηματίζει πυκνά νέφη ενώ σε θερμοκρασίες άνω των $1000 K$ είναι διάχυτο και αραιό.
Οι κύριοι τρόποι παρατήρησης του μεσοαστρικού αερίου είναι οι:

\paragraph{Παρατηρήσεις ατομικού και μοριακού Υδρογόνου}
\paragraph{Παρατηρήσεις του Ιονισμένου Υδρογόνου}
Στη περίπτωση

\section{Μοριακά Νέφη}

\printbibliography

\end{document}