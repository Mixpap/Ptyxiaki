\documentclass[a4paper,11pt]{memoir}
\usepackage[greek]{babel}
%\usepackage[iso-8859-7]{inputenc}
\usepackage[utf8x]{inputenx}
\usepackage[LGR,T1]{fontenc}
\usepackage{latexsym,graphicx}
\usepackage{cite}
\usepackage{amsmath}
\usepackage{todonotes}
\input epsf

\newcommand{\n}[1]{\todo[size=\tiny]{#1}}
\newcommand{\dd}{\todo[size=\tiny]{διατύπωση/μεταφρασγ}}
\newcommand{\bb}{\todo[size=\tiny]{βιβλιογραφία}}
\newcommand{\sm}{$M_{\odot}$}

\begin{document}
\chapter{Ύλη μεταξύ των αστέρων}
Στον μέσοαστρικό χώρο υπάρχει μια τεράστια ποσότητα ύλης \n{ποσοστό στο γαλαξία?} υπο μορφή αερίου και σκόνης. Το υλικό αυτό και τα επιμέρους συστατικά του είναι η πρωτογενής αιτία \dd της δημιουργίας των αστέρων άρα η έρευνα για τη σύνθεσης και τα χαρακτηριστικά τους είναι απαραίτητη για την βαθύτερη κατανόηση της πρώιμης \dd δημιουργίας των αστέρων.

Σήμερα γνωρίσουμε οτι η ύλη μεταξύ των αστέρων αποτελείται περίπου κατα 99\% απο αέριο και κατα 1\% απο σκόνη με τη συνολική της μάζα στο γαλαξία μας είναι της τάξης των \sm \n{μάζα αερίου} ενώ η πυκνότητα της κυμαίνεται απο $10^{-4}$ έως $10^{6}$ σωματίδια ανά $cm^3$ \bb.

\section{Το μεσοαστρικό αέριο}
Από τις πρώτες κιόλας παρατηρήσεις του μεσοαστρικού αερίου παρατηρήσαμε ότι βρίσκεται κυρίως σε μορφή διακριτών συμπυκνώσεων δηλαδή έχουν δομή νεφών ενώ μπορεί να βρίσκεται σε μορφή ατομική, ιονισμένη και μοριακή ανάλογως τη θερμοκρασία του περιβάλλοντος, την πυκνότητα και το radiative enviroment \dd. Τα συστατικά του αερίου είναι:
\begin{itemize}
 \item Υδρογόνο $(H_2,H,H I,H II, e^-)$
 \item Ήλιο $(He I,He II)$
 \item Trace Elements $(C,O,Ne,Mg,Fe, \text{ κ.α.})$ \dd, 
 \item Μόρια $(CO,Cs,\text{ κ.α.})$ 
\end{itemize}

\subsection{Παρατηρήσεις ατομικού και μοριακού Υδρογόνου}

 
 
 \end{document}