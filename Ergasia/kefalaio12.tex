\documentclass[a4paper,11pt]{memoir}

\usepackage{fontspec}
\usepackage{float}

\usepackage{polyglossia}
\setmainlanguage{greek}
\setotherlanguage{english}
\setmainfont{Liberation Serif}
%\newfontfamily\greekfont{Liberation Serif}
%\newfontfamily\greekfontsf{Liberation Serif}
%\newfontfamily\greekfonttt{Liberation Serif}

\usepackage{latexsym,graphicx}
\usepackage{todonotes}
\usepackage[version=3]{mhchem}
%\usepackage[style=nature]{biblatex}
\usepackage{biblatex}
\bibliography{Bibliography.bib}

%====================Custom Commands========================
%\newcommand{\n}[1]{\todo[size=\tiny]{#1}}
\newcommand{\sm}{$M_{\odot}$}
%===========================================================


%================Memoir Options============================
\setcounter{secnumdepth}{3} %Depth of numbering 3=subsubsection
%memoir page style, ligo allagmeno
%\copypagestyle{myruled}{ruled}
%\makeevenhead{myruled}{\footnotesize\slshape\leftmark}{}{}
%\makeoddhead{myruled}{}{}{\footnotesize\slshape\rightmark}
%\pagestyle{myruled}


%\chapterstyle{ger}
%\chapterstyle{lyhne}
%\chapterstyle{madsen}
%\renewcommand*{\chapnumfont}{\normalfont\huge\bfseries}
%\renewcommand*{\chaptitlefont}{\normalfont\huge\bfseries}
%============================================================



\begin{document}

\chapter{Μοριακά Νέφη και η Ύλη μεταξύ των Αστέρων}

Στον μεσοαστρικό χώρο υπάρχει μια τεράστια ποσότητα ύλης \todo{ποσοστό στο γαλαξία?} υπό τη μορφή αερίου και σκόνης. Το υλικό αυτό είναι η πρωτογενής αιτία \todo{διατύπωση} της δημιουργίας των αστέρων άρα η έρευνα για τη σύνθεση και τα χαρακτηριστικά της είναι απαραίτητη για την βαθύτερη κατανόηση της πρώιμης \todo{διατύπωση} δημιουργίας των αστέρων.

Σήμερα γνωρίσουμε οτι η ύλη μεταξύ των αστέρων αποτελείται περίπου κατα 99\% απο αέριο και κατα 1\% απο σκόνη με τη συνολική της μάζα στο γαλαξία μας να είναι της τάξης των \sm \todo{μάζα αερίου} ενώ η πυκνότητα της κυμαίνεται απο $10^{-4}$ έως $10^{6}$ σωματίδια ανά $cm^3$.

\paragraph{Μεσοαστρικό Αέριο} 
Το Μεσοαστρικό Αέριο παρατηρείται κυρίως σε μορφή διακριτών συμπυκνώσεων δηλαδή έχουν δομή νεφών ενώ μπορεί να βρίσκεται σε μορφή ατομική, ιονισμένη και μοριακή.  
Τα συστατικά του αερίου είναι:
\begin{itemize}
	\item Υδρογόνο (\ce{H2}, \ce{H}, \ce{H I}, \ce{H II}, \ce{e-})
	\item Ήλιο (\ce{He I},\ce{He II})
	\item Trace Elements \todo{Μετάφραση} (\ce{C},\ce{O},\ce{Ne},\ce{Mg},\ce{Fe}, κ.α.)
	\item Μόρια (\ce{CO},\ce{Cs}, κ.α.) 
\end{itemize}
με το κυριότερο σε αναλόγια \todo{αναλογία Υδρογόνου} να είναι το Υδρογόνο.

\section{Φάσεις και χαρακτηριστικά της Μεσοαστρικής Ύλης}
Η Μεσοαστρική Ύλη (ISM) απαντάται σε τρεις φάσεις με διαφορετικά φυσικά και χημικά χαρακτηριστικά: 
\footnote{Για τα χημικά χαρακτηριστικά αναφερόμαστε στή σύνθεση των μορίων και στην αναλογία των στοιχείων. Στα φυσικά χαρακτηριστικά αναφερόμαστε στη πυκνότητα και τη θερμοκρασία της Ύλης} 
τη \textbf{ψυχρή} που αποτελείται απο μοριακό και ατομικό αέριο Υδρογόνου και σκόνη, τη \textbf{θερμή} από ατομικό και ιονισμένο άεριο Υδρογόνο και την \textbf{υπέρθερμη} από διεγερμένο αέριο από κρουστικά κύματα εκρήξεων supernova.


\subsection{Ενεργειακή ισορροπία}
\label{par:EnergyBalance}
Η κινητική θερμοκρασία \footnote{Το ψυχρό μεσοαστρικό αέριο λόγω της γενικά χαμηλής του πυκνότητας δεν βρίσκεται σε θερμοδυναμική ισορροπία. Επομένως όταν μιλάμε για θερμοκρασία αναφερόμαστε στη κινητική του θερμοκρασία.\cite[p. 28]{spitzer_physical_1998}} της Μεσοαστρικής Ύλης κυμαίνεται σε ένα εύρος τιμών 6 τάξεων μεγέθους όπως παρατηρούμε και από τον πίνακα~\ref{tab:ISM}. Για να περιγράψουμε και να μοντελοποιήσουμε την ενεργειακή ισορροπία στη Μεσοαστρική Ύλη άρα να εξηγήσουμε και τις παρατηρούμενες θερμοκρασίες θα πρέπει να υπολογίσουμε τις διαδικασίες θέρμανσης και ψύξης. 
Η κύρια διαδικασία ψύξης είναι η εκπομπή ακτινοβολίας είτε μέσω αυθόρμητης αποδιέγερσης ή αποδιέγερσης λόγω κρούσης. Ενώ για τη θέρμανση έχουμε μια πληθώρα διαδικασιών θέρμανσης οι οποίες μπορούν να ταξινομηθούν σε 3 κατηγορίες:
\begin{itemize}
	\item θέρμανση από πεδία ακτινοβολίας: φωτοηλεκτρική απορρόφηση σε ουδέτερα στοιχεία, φωτοδιάσπαση στα μόρια, φωτοιονισμός.
	\item θέρμανση μέσω συγκρούσεων: από τυρβώδες ροές, κρουστικά κύματα καταλοίπων supernova και κοσμικής ακτινοβολίας.
	\item θερμική ανταλλαγή μεταξύ της σκόνης και νεφών αερίου, αλληλεπίδραση ιονισμένου αερίου με μαγνητικά πεδία, βαρυτική κατάρρευση. 
\end{itemize}

\subsection{Παρατηρήσεις της Μεσοαστρικής Ύλης}
Η παρατήρηση και μελέτη της Μεσοαστρικής Ύλης ποικίλει αναλόγως τη φάση στην οποία βρίσκεται.
\subparagraph{Παρατήρηση 21.1 cm}
H καλύτερη μέχρι σήμερα δυνατή μέθοδος για την παρατήρηση του \textbf{Ουδέτερου Υδρογόνου \ce{H I}} είναι η εκπομπή της γραμμής $21.1 \, cm$ στα ραδιοκύματα που οφείλεται στη μετάπτωση αντιστροφής του spin του πρωτονίου και του ηλεκτρονίου στη βασική κατάσταση του ατόμου του Υδρογόνου. Η ενεργειακή διαφορά των καταστάσεων με συνολικό spin $F=1$ \textbf{(τα spin $p^+$ και $e^-$ είναι παράλληλα)} και $F=0$ \textbf{(τα spin $p^+$ και $e^-$ είναι άντιπαράλληλα)} είναι $h \nu=6\times 10^{-6} \, eV$ η οποία αντιστοιχεί στη γραμμή των 21 cm.
Ο συντελεστής Einstein για την αυθόρμητη εκπομπή είναι $A_{10} \simeq 3\times 10^{-15}s^{-1}$ που αντιστοιχεί σε μια χρονική κλίμακα των $10^7$ ετών στην οποία παραμένει ένα διεγερμένο άτομο Υδρογόνου μέχρι να αποδιεγερθεί αυθόρμητα εκπέμποντας το παρατηρούμενο φωτόνιο. Ο πολύ μικρός αυτός ρυθμός εκπομπής αντιπαραβάλλεται \todo{διατύπωση} εν τέλει από τη τεράστια ποσότητα του ατομικού υδρογόνου έτσι ώστε να είναι \todo{ολοκλήρωση}    
\todo{φάσματα απορρόφησης}

\subparagraph{Περιοχές \ce{H\alpha}}



\begin{table}
	\caption{Χαρακτηριστικά της μεσοαστρικής ύλης και περιοχές παρατήρησης}
	\label{tab:ISM}
	\begin{tabular}{p{2.6cm} p{2.5cm}  c  c  p{4.75cm}}
		\toprule
		Κατηγορία & Κατάσταση Υδρογόνου & $ T \,(K)$ & $ n \,(cm^{-1})$ & Περιοχή Παρατηρήσεων \\ \hline
		Μοριακά Νέφη & Μοριακό \ce{H2} & 10-50 & $>10^3$ & Μοριακή εκπομπή - απορρόφηση στο Ράδιο και στο Υπέρυθρο \\
		Ψυχρά Νέφη \ce{H I} & Ατομικό \ce{H} & $100$ & $30$ & Γραμμή απορρόφησης $21 \,cm$\\
		Θερμό \ce{H I} & Ατομικό \ce{H} & $10^3$ & $0.1$ & Γραμμή εκπομπής $21 \,cm$\\
		Θερμό \ce{H IΙ} & Ιονισμένο \ce{H+} & $10^4$ & $10^{-2}$ & Γραμμή Εκπομπής \ce{H\alpha}\\
		Περιοχές \ce{H IΙ} & Ιονισμένο \ce{H+}& $10^4$ & $>100$ & Γραμμή Εκπομπής \ce{H\alpha}\\
		Υπέρθερμο Ιονισμένο αέριο & Ιονισμένο \ce{H+}& $10^6-10^7$ & $10^{-3}$ & Εκπομπή ακτινοβολίας Χ, Απορρόφηση από ιονισμένα μέταλλα\\
		\bottomrule
	\end{tabular}
\end{table}

	
\section{Μοριακά Νέφη}
Οί πιο ενδιαφέρουσες -σχετικά με τη δημιουργία αστέρων- περιοχές του Μεσοαστρικού Υλικού είναι αυτές όπου εμφανίζεται πυκνότερο (10-30 άτομα ανά $cm^{3}$) και βρίσκεται στη \textbf{ψυχρή φάση} (τυπικές θερμοκρασίες $10-50 \, K$) δηλαδή \todo{διατύπωση} έχει νεφελώδη μορφή και απαρτίζεται κυρίως από μοριακό και ατομικό Υδρογόνο. Οι περιοχές αυτές ονομάζονται Μοριακά Νέφη (Molecular Clouds).

\subsection{Χαρακτηριστικά των Μοριακών Νεφών}
Όπως αναφέραμε γενικότερα στη παράγραφο~\ref{par:EnergyBalance} η θερμοκρασία ενός νέφους είναι αποτέλεσμα στης ενεργειακής ισορροπίας μεταξύ των μηχανισμών θέρμανσης και ψύξης. Για τα Μοριακά Νέφη συγκεκριμένα η θέρμανση είναι αποτέλεσμα της θερμότητας που παρέχεται από κοντινά άστρα ή μέσω της κοσμικής ακτινοβολίας, ενώ η ψύξη επιτυγχάνεται μέσω διαδικασιών απορρόφησης και κρούσης με τα σωματίδια της σκόνης ή του αερίου.
Η ενέργεια τελικά αποδίδεται μέσω της εκπομπής υπέρυθρης ακτινοβολίας.

\subsubsection{Δημιουργία του Μοριακού Υδρογόνου}
Το κυριότερο συστατικό των Μοριακών Νεφών είναι το μοριακό Υδρογόνο (\ce{H2}). Όταν δύο άτομα Υδρογόνου ενώνονται και δημιουργούν ένα μόριο \ce{H2} αυτό κερδίζει ενέργεια η οποία δεν μπορεί να αποδοθεί στο περιβάλλον με αποτέλεσμα το μόριο να διασπάται. Αν όμως η διαδικασία αυτή γίνει πάνω σε έναν κόκκο σκόνης, τότε αυτός λειτουργεί καταλυτικά απορροφώντας το πλεόνασμα ενέργειας και το μόριο παραμένει σταθερό. Έτσι πραγματοποιείται μια διαδικασία ανάδρασης όπου τα άτομα Υδρογόνου τροφοδοτούν περιοχές μεγάλης πυκνότητας με \ce{H2} αυξάνοντας έτσι κι άλλο τη τοπική πυκνότητα. Αποτέλεσμα είναι να δημιουργούνται περιοχές μεγάλης πυκνότητας όπου λαμβάνει χώρα και η δημιουργία των αστέρων.

\subsubsection{Μορφολογία Μοριακών Νεφών}
Η παραπάνω διαδικασία δίνει στα μοριακά νέφη μια ιεραρχικά δομημένη μορφή όπου οι πυκνότερες περιοχές έχουν μικρότερη κλίμακα μήκους (clumpiness). Εκτός από τη clumpiness μορφολογία -που φαίνεται και από το πίνακα~\ref{tab:MCtypes}, παρατηρούμε και νηματώδεις (filaments) δομές συμπυκνώσεων.

\begin{table}
	\caption{Χαρακτηριστικά και διαφορετικοί τύποι Μοριακών Νεφών}
	\label{tab:MCtypes}
	\begin{tabular}{l c  c  c  r}
		\toprule
		Κατηγορία & Μέση ακτίνα (pc) & $ T \,(K)$ & $ n(H_2) \,(cm^{-3})$ & Μάζα (\sm) \\ \hline
		Γιγαντιαίο Μοριακό Νέφος & $20$ & $15$ & $100$ & $10^5$ \\
		Μοριακό Νέφος & $5$ & $10$ & $300$ & $10^4$\\
		clump & $2$ & $10$ & $10^3$ & $10^3$\\
		Πυρήνας Νέφους & $0.08$ & $10$ & $10^5$ & $10$\\
		\bottomrule
	\end{tabular}
\end{table}


\section{Παρατηρήσεις των Μοριακών Νεφών}
\label{par:H2}
Παρά τη "κυριαρχία" του μοριακού υδρογόνου στα Μοριακά Νέφη είναι αδύνατον να το παρατηρήσουμε καθώς η ενεργειακή διαφορά ακόμα και των χαμηλότερων διεγερμένων από τη βασική στάθμη είναι πολύ μεγάλη. Έτσι στις χαμηλές θερμοκρασίες των Μοριακών Νεφών, η μόνη δυνατότητα να παρατηρήσουμε άμεσα το \ce{H2} είναι μέσω γραμμών απορρόφησης από πηγές στο υπόβαθρο. Ο εναλλακτικός τρόπος παρατήρησης του \ce{H2} είναι εμμέσως μέσω της εκπομπής διαφορετικών μορίων που είναι πιο "ευαίσθητα" στις χαμηλές θερμοκρασίες, όπως του Μονοξειδίου του Άνθρακα (\ce{12CO}) και των ισοτόπων του (\ce{13CO},\ce{C18O), της αμμωνίας (\ce{NH3}) και άλλων (\ce{CS},\ce{H2CO},\ce{H2O},\ce{OH}).
Γνωρίζοντας την αναλογία μεταξύ των μορίων μπορούμε να υπολογίσουμε τη ποσότητα του \ce{H2}.

\subsection{Ενεργειακές μεταβάσεις του \ce{H2}}
Το \ce{H2} είναι ένα πλήρως συμμετρικό μόριο άρα δεν έχει μόνιμη διπολική ροπή. Άρα αφού οι μεταβάσεις του ηλεκτρικού διπόλου είναι απαγορευμένες οι επόμενες είναι οι τετραπολικές. 
Η ενέργεια περιστροφής είναι $E_{rot}=\frac{h^2}{2I_{H_2}}J(J+1)$ όπου $J$ ο περιστροφικός κβαντικός αριθμός και $I_{H_2}=5\times 10^{-48} \, kg\, m^2$ η ροπή αδράνειας του \ce{H2}.
Για τις τετραπολικές μεταβάσεις έχουμε ότι $\Delta J =0,\pm 2$, άρα για το \ce{H2} αυτό μπορεί να βρίσκεται σε δύο μορφές, αυτή του παρά-\ce{H2} όπου είναι κατειλημμένες μόνο οι καταστάσεις με $J=0,2,4,6,...$ και η όρθο-\ce{H2} όπου είναι κατειλημμένες μόνο οι καταστάσεις με $J=1,2,5,...$. 
Άρα η χαμηλότερη ενεργειακή διαφορά από τη βασική κατάσταση ($J=0$) είναι η 
$$
\Delta E=E(J=2)-E(J=2)\simeq 7.5\times 10^{-21}\, J
$$
η οποία αντιστοιχεί σε θερμοκρασία $510 \,K$. Η αυθόρμητη αποδιέγερση έχει συντελεστή Einstein $A_{20}=3\times 10^{-11} \, s^{-1}$ και παράγει ένα φωτόνιο μήκους κύματος $28.2\, \mu m$ στο υπέρυθρο.

Αν εργαστούμε αντίστοιχα για τις ταλαντωτικές μεταβάσεις, βρίσκουμε ότι αυτές αντιστοιχούν σε θερμοκρασίες χιλιάδων βαθμών κέλβιν. Για τέτοιες θερμοκρασίες ένα διεγερμένο μόριο \ce{H2} φτάνει στη βασική του κατάσταση με συνδυασμό ταλαντωτικών και περιστροφικών μεταβάσεων. Οι εκπομπές αυτές είναι χαρακτηριστικές στα μέτωπα κυμάτων κρούσης όπου το \ce{H2} θερμαίνεται σε χιλιάδες βαθμούς κέλβιν.

\subsection{Παρατηρήσεις στο \ce{CO} }
Εφόσον το \ce{H2} είναι δύσκολο να το παρατηρήσουμε χρησιμοποιούμε το Μονοξείδιο του Άνθρακα \ce{CO} σαν tracer του μοριακού αερίου. Το \ce{CO} είναι το δεύτερο σε αναλογία μόριο στο Σύμπαν (μετά το \ce{H2}) και έχει μόνιμη διπολική ροπή άρα έχουμε περιστροφικές ενεργειακές μεταβάσεις με $\Delta J=\pm 1$ το οποίο του επιτρέπει να εκπέμπει σημαντικά στο ραδιοφωνικό φάσμα. 
Σε αντιστοιχία με τη διαδικασία που κάναμε στη παράγραφο~\ref{par:H2} βρίσκουμε για το \ce{CO} για τη χαμηλότερη μετάβαση $J=1\rightarrow 0$ $\Delta E=4.8\times 10^{-4} eV$ το οποίο αντιστοιχεί σε θερμοκρασία $5.5 \, K$. Η μετάβαση αυτή αποδίδει ένα ραδιοφωνικό φωτόνιο στα $2.6 \, mm$ και ο συντελεστής Einstein για την αυθόρμητη αποδιέγερση είναι $Α_{10}=7.5\times 10^{-8} \, s^{-1}$.
Ο κύριος μηχανισμός διέγερσης ενός μορίου \ce{CO} στη $J=1$ είναι μέσω της σύγκρουσης του με ένα μόριο \ce{H2}. Αφού διεγερθεί η αποδιέγερση του μπορεί να γίνει είτε εκπέμποντας ένα φωτόνιο στα $2.6 \, mm$ σε περιοχές με χαμηλή συνολική πυκνότητα είτε μεταφέροντας την ενέργεια του σε ξανά σε ένα μόριο \ce{H2} χωρίς να εκπεμφθεί φωτόνιο σε περιοχές με μεγάλη συνολική πυκνότητα. Για να βρούμε τη κρίσιμη πυκνότητα όπου διαχωρίζονται αυτές οι δύο περιοχές θεωρούμε ότι η πιθανότητα αυθόρμητης εκπομπής $A_{ij}$ της μετάβασης $i\rightarrow j$ είναι ίση με τη πιθανότητα εκπομπής λόγω σύγκρουσης $n \, \gamma _{ij}$. Άρα η κρίσιμη πυκνότητας είναι:
$$
n_{crit}=\frac{A_{ij}}{\gamma _{ij}}
$$ 
Για μια τυπική θερμοκρασία $T=10 \, K$ βρίσκουμε $n_{crit}=3\times 10^3 \,cm^{-3}.

\section{IMF}











\printbibliography

\end{document}