Στον μεσοαστρικό χώρο υπάρχει μια τεράστια ποσότητα ύλης \todo{ποσοστό στο γαλαξία?} υπό τη μορφή αερίου και σκόνης. Το υλικό αυτό είναι η πρωτογενής αιτία \todo{διατύπωση} της δημιουργίας των αστέρων άρα η έρευνα για τη σύνθεση και τα χαρακτηριστικά της είναι απαραίτητη για την βαθύτερη κατανόηση της πρώιμης \todo{διατύπωση} δημιουργίας των αστέρων.


Σήμερα θεωρύμε σε γενικές γραμμές ότι η ύλη μεταξύ των αστέρων αποτελείται περίπου κατα 99\% απο αέριο και κατα 1\% απο σκόνη με τη συνολική της μάζα στο γαλαξία μας να είναι της τάξης των \sm \todo{μάζα αερίου} ενώ η πυκνότητα της κυμαίνεται απο $10^{-4}$ έως $10^{6}$ σωματίδια ανά $cm^3$.


Το Μεσοαστρικό Αέριο παρατηρείται σε νεφελώδη μορφή και αποτελείται κυρίως (περίπου το 90\%) από υδρογόνο σε ατομική, ιονισμένη και μοριακή κατάσταση. Δεύτερο σε αναλογία είναι το Ήλιο (περίπου 9\%) ενώ το υπόλοιπο 1\% είναι βαρύτερα στοιχεία (\ce{C},\ce{O},\ce{Ne},\ce{Mg},\ce{Fe}, κ.α.) και μόρια (\ce{CO},\ce{CS}, κ.α.).
Τα μόρια 


Η Μεσοαστρική Σκόνη αποτελείται κυρίως από άνθρακα και πυρίτιο σε ενώσεις με Υδρογόνο, Οξυγόνο, Μαγνήσιο και Σίδηρο ενώ το μέγεθος των κόκκων της σκόνης κυμαίνεται από \SI{0.01}{\micro\meter} έως \SI{1}{\micro\meter} ακολουθώντας μια κατανομή δύναμης όπου τα μικρότερα μεγέθη είναι πολυπληθέστερα από τα μεγαλύτερα. 
Η Μεσοαστρική Σκόνη παρατηρείται στις σπείρες του Γαλαξία μας (αλλά και σε άλλους γαλαξίες) με τη χαρακτηριστική μορφή τεράστιων σκοτεινών "δρόμων" λόγο της επισκότισης των όπισθεν αστέρων λόγω της απορρόφησης και σκέδασης του ορατού φωτός.
