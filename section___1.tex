\section{Παρατηρήσεις των Μοριακών Νεφών}
\label{par:H2} 

Παρά τη "κυριαρχία" του μοριακού υδρογόνου στα Μοριακά Νέφη είναι απίθανο να το παρατηρήσουμε καθώς η ενεργειακή διαφορά ακόμα και των χαμηλότερων διεγερμένων από τη βασική του στάθμη είναι πολύ μεγάλη, όπως θα δείξουμε παρακάτω. Έτσι στις χαμηλές θερμοκρασίες των Μοριακών Νεφών, η μόνη δυνατότητα να παρατηρήσουμε άμεσα το \ce{H2} είναι μέσω γραμμών απορρόφησης από πηγές στο υπόβαθρο \footnote{μέσω των γραμμών απορρόφησης στο Υπεριώδες}. 


Ο εναλλακτικός τρόπος παρατήρησης του \ce{H2} είναι εμμέσως μέσω της εκπομπής διαφορετικών μορίων που είναι πιο "ευαίσθητα" στις χαμηλές θερμοκρασίες, όπως του Μονοξειδίου του Άνθρακα (\ce{^{12}CO}) και των ισοτόπων του (\ce{^{13}CO}, \ce{C^{18}O}), της αμμωνίας (\ce{NH3}) και άλλων (\ce{CS}, \ce{H2CO}, \ce{H2O}, \ce{OH}).
Γνωρίζοντας την αναλογία μεταξύ των μορίων μπορούμε να υπολογίσουμε τη ποσότητα του \ce{H2}.


Εκτός από τη παρατήρηση της μοριακής συνιστώσας του νέφους, έχουμε στη διάθεση μας και άλλες περιοχές παρατήρησης όπως η εκπομπή των κόκκων σκόνης στο Υπέρυθρο και η εξάλειψη από τους ίδιους του ορατού φώτος αστέρων του υποβάθρου.


\subsection{Ενεργειακές μεταβάσεις του \ce{H2}}
Το \ce{H2} είναι ένα πλήρως συμμετρικό μόριο άρα δεν έχει μόνιμη διπολική ροπή. Άρα καθώς οι μεταβάσεις του ηλεκτρικού διπόλου είναι απαγορευμένες οι επόμενες είναι οι τετραπολικές. 
Η ενέργεια περιστροφής είναι $E_{rot}=\frac{h^2}{2I_{H_2}}J(J+1)$ όπου $J$ ο περιστροφικός κβαντικός αριθμός και $I_{H_2}=5\times 10^{-48} \, kg\, m^2$ η ροπή αδράνειας του \ce{H2}.
Για τις τετραπολικές μεταβάσεις έχουμε ότι $\Delta J =0,\pm 2$, άρα για το \ce{H2} αυτό μπορεί να βρίσκεται σε δύο μορφές, αυτή του παρά-\ce{H2} όπου είναι κατειλημμένες μόνο οι καταστάσεις με $J=0,2,4,6,...$ και η όρθο-\ce{H2} όπου είναι κατειλημμένες μόνο οι καταστάσεις με $J=1,2,5,...$. 
Άρα η χαμηλότερη ενεργειακή διαφορά από τη βασική κατάσταση ($J=0$) είναι η 
\begin{equation}
\Delta E=E(J=2)-E(J=0)\simeq 4.7\times 10^{-2}\, eV
\end{equation}


η οποία αντιστοιχεί σε θερμοκρασία $510 \,K$. Από τη μετάβαση παράγεται ένα ένα φωτόνιο μήκους κύματος $28.2\, \mu m$ στο υπέρυθρο ενώ ο συντελεστής Einstein είναι $A_{20}=3\times 10^{-11} \, s^{-1}$.


Αν εργαστούμε αντίστοιχα για τις ταλαντωτικές μεταβάσεις, βρίσκουμε ότι αυτές αντιστοιχούν σε θερμοκρασίες χιλιάδων βαθμών κέλβιν. Για τέτοιες θερμοκρασίες ένα διεγερμένο μόριο \ce{H2} φτάνει στη βασική του κατάσταση με συνδυασμό ταλαντωτικών και περιστροφικών μεταβάσεων. Οι εκπομπές αυτές είναι χαρακτηριστικές στα μέτωπα κυμάτων κρούσης όπου το \ce{H2} θερμαίνεται σε πολύ υψηλές θερμοκρασίες.


Άρα για τις τυπικές θερμοκρασίες των μοριακών νεφών $10-50\ K$ είναι αδύνατον να το παρατηρήσουμε άμεσα. Αντί αυτού χρησιμοποιούμε το επόμενο σε αναλογία μόριο, το \ce{CO} και θεωρώντας γνωστή την αναλόγια του με το \ce{H2}, υπολογίζουμε το δεύτερο. 


\subsection{Παρατηρήσεις στο \ce{CO} }
Εφόσον το \ce{H2} είναι δύσκολο να το παρατηρήσουμε χρησιμοποιούμε το Μονοξείδιο του Άνθρακα \ce{CO} σαν tracer \todo{μετάφραση} του μοριακού αερίου. Το \ce{CO} είναι το δεύτερο σε αναλογία μόριο στο Σύμπαν (μετά το \ce{H2}) και έχει μόνιμη διπολική ροπή άρα έχουμε περιστροφικές ενεργειακές μεταβάσεις με $\Delta J=\pm 1$ το οποίο του επιτρέπει να εκπέμπει σημαντικά στο ραδιοφωνικό φάσμα. 
Σε αντιστοιχία με τη διαδικασία που κάναμε στη παράγραφο~\ref{par:H2} βρίσκουμε για το \ce{CO} για τη χαμηλότερη μετάβαση $J=1\rightarrow 0$ $\Delta E=4.8\times 10^{-4} eV$ το οποίο αντιστοιχεί σε θερμοκρασία $5.5 \, K$. Η μετάβαση αυτή αποδίδει ένα ραδιοφωνικό φωτόνιο στα $2.6 \, mm$ και ο συντελεστής Einstein για την αυθόρμητη αποδιέγερση είναι $A_{10}=7.5\times 10^{-8} \, s^{-1}$.


Ο κύριος μηχανισμός διέγερσης ενός μορίου \ce{CO} στη $J=1$ είναι μέσω της σύγκρουσης του με ένα μόριο \ce{H2}. Αφού διεγερθεί η αποδιέγερση του μπορεί να γίνει είτε εκπέμποντας ένα φωτόνιο στα $2.6 \, mm$ σε περιοχές με χαμηλή συνολική πυκνότητα είτε μεταφέροντας την ενέργεια του σε ξανά σε ένα μόριο \ce{H2} χωρίς να εκπεμφθεί φωτόνιο σε περιοχές με μεγάλη συνολική πυκνότητα. Για να βρούμε τη κρίσιμη πυκνότητα όπου διαχωρίζονται αυτές οι δύο περιοχές θεωρούμε ότι η πιθανότητα αυθόρμητης εκπομπής $A_{ij}$ της μετάβασης $i\rightarrow j$ είναι ίση με τη πιθανότητα εκπομπής λόγω σύγκρουσης $n \, \gamma _{ij}$. Άρα η κρίσιμη πυκνότητας είναι:
\begin{equation}
n_{crit}=\frac{A_{ij}}{\gamma _{ij}}
\end{equation}


Για μια τυπική θερμοκρασία $T=10 \, K$ βρίσκουμε $n_{crit}=3\times 10^3 \,cm^{-3}$.


Σε αυτό το κεφάλαιο θα παρουσιάσουμε τις κυριότερες θεωρίες δημιουργίας πρωτοαστέρων μέσα στους πυρήνες των μοριακών νεφών και την επίδραση τους στο περιβάλλον του μοριακού νέφους.
