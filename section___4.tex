\section{Δίσκοι Προσαύξησης}
Αναφερθήκαμε προηγουμένων πως ένα αρχικά περιστρεφόμενο μοριακό νέφος θα δημιουργήσει κατά τη κατάρρευση του ένα δίσκο προσαύξησης γύρω από τον πρωτοαστέρα. Σε αυτή τη φάση ο πρωτοαστέρας έχει μάζα μόλις $10^{-2}$ \sm, κάτι το οποίο σημαίνει ότι μάζα προσπίπτει στον αστέρα μέσω του δίσκου.


Αν θεωρήσουμε ότι ένας στοιχειώδης δακτύλιος πάχους $\delta r$ του δίσκου σε απόσταση $r$ από πρωτοαστέρα μάζας $M$ κινείται με κεπλεριανή ταχύτητα $v_{\phi}=\sqrt{\frac{GM}{r}}$ τότε η ειδική στροφορμή του θα είναι $r \, v_{\phi}=\sqrt{GMr}$ δηλαδή θα αυξάνεται με την απόσταση. Άρα για να καταφέρει ο δακτύλιος αυτός να φτάσει το πρωτοαστέρα θα πρέπει η στροφορμή του συνεχώς να ελαττώνεται, δηλαδή να υπάρχει κάποιος μηχανισμός ο οποίος μέσω κάποιας ροπής δύναμης θα οδηγεί σε απώλεια στροφορμής.


Οι πιο πιθανοί τέτοιοι μηχανισμοί είναι: η εσωτερική τριβή του ρευστού του δίσκου, η επίδραση του ενεργού ιξώδες λόγω τυρβώδους ροής και η απώλεια στροφορμής μέσω μαγνητισμένων εκροών.
Με τον τελευταίο μηχανισμό θα ασχοληθούμε στις επόμενες παραγράφους.
