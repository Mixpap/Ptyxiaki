\section{Η επίδραση των πρωτοαστέρων στο περιβάλλον τους}


Από τις πρώτες στιγμές της δημιουργίας τους οι αστέρες επηρεάζουν σημαντικά το περιβάλλον τους μέσω δυναμικών μηχανισμών όπως οι εκροές μάζας (Outflows, Jets) και ο αστρικός άνεμος αλλά και λόγω της επίδρασης της ακτινοβολίας (ιονισμός Υδρογόνου). Η επίδραση αυτών των μηχανισμών στο περιβάλλον τους έχει πολύ μεγάλη εξάρτηση από τη μάζα των πρωτοαστέρων, καθώς οι αστέρες μεγάλης μάζας έχουν πολύ ισχυρότερη επιρροή από τους μέσης ή μικρής μάζας.


Συγκεκριμένα οι μηχανισμοί εκροών μάζας εμφανίζονται σε όλους τους πρωτοαστέρες ανεξαρτήτως μάζας τηρουμένων των αναλογιών. Οι μηχανισμοί αυτοί είναι απόροια της διαδικασίας προσαύξησης των πρωτοαστέρων και άρα έχουν περιορισμένο χρόνο ζωής με αποτέλεσμα η επίδραση τους να γίνεται αισθητή μόνο στο άμεσα γειτονικό περιβάλλον \todo{μεχρι που?}. 


Απεναντίας οι αστρικοί άνεμοι, και τα φωτόνια υψηλών ενεργειών που δημιουργούνται μόνο στους αστέρες μεγάλης μάζας, επιδρούν στο περιβάλλον για πολύ μεγαλύτερο χρονικό διάστημα με καταστροφικές συνέπειες για πολύ μεγάλο μέρος του μοριακού νέφους.


\subsection{Πίδακες και εκροές υλικού}
Οι κυρίαρχες θεωρίες για την δημιουργία των πιδάκων υλικού (jets) βασίζονται στο συνδυασμό της περιστροφικής κίνησης του δίσκου και του αστέρα και στο διπολικό μαγνητικό πεδίο του αστέρα. 
Σύμφωνα με τα υπάρχουσα μοντέλα υλικό υπό τη μορφή αστρικού ανέμου από το δίσκο ευθυγραμμίζεται και επιταχύνεται κάθετα στο δίσκο και στη διεύθυνση του άξονα περιστροφής δημιουργώντας μια σχετικά στενή δομή που διατηρείται για αρκετά μεγάλες αποστάσεις.


Οι πίδακες τροφοδοτούν με ενέργεια όχι μόνο το άμεσο περιβάλλον του πρωτοαστέρα αλλά και αέριο του μοριακού νέφους πέρα από τον αρχικό πυρήνα. Η ενέργεια αυτή δημιουργεί τύρβη στο νέφος


\subsection{Αστέρες μεγάλης μάζας}
Οι μεγάλης μάζας αστέρες είναι αστέρες των οποίων οι μάζες ξεπερνούν τις 8 \sm με φασματικούς τύπους O και B, χαρακτηρίζονται από ταχύτατους ρυθμούς εξέλιξης \footnote{Οι μεγάλης μάζας αστέρες ξεκινούν τη καύση του Υδρογόνου ενώ βρίσκονται ακόμα στη φάση της προσαύξησης, ενώ και ο χρόνος ζωής τους δεν ξεπερνάει τα $3 \times 10 ^7 yr$} με τεράστιες λαμπρότητες ($L_* > 10^4 L _ \odot$) και επιφανειακή θερμοκρασία $>10 ^5 \ K$.


\subsubsection{Περιοχές HII}
\label{par:HII regions} 

Λόγω των πολύ υψηλών θερμοκρασιών, οι OB αστέρες εκπέμπουν υψηλό αριθμό φωτονίων υψηλών ενεργειών (μεγαλύτερες από το όριο Lyman, στο υπεριώδες) τα οποία διασπούν το μοριακό υδρογόνο σε δύο ατομικά τα οποία τελικά θα ιονιστούν. Οι περιοχές όπου το αέριο υδρογόνο είναι ιονισμένο ονομάζονται περιοχές HII. Στις περιοχές $HII$ το πλάσμα υδρογόνου επιχειρεί συνεχώς να επανασυνδεθεί για να σχηματίσει ουδέτερα άτομα υδρογόνου αλλά εμποδίζεται από τη συνεχιζόμενη παραγωγή υπεριωδών φωτονίων.


Μπορούμε να ορίσουμε μια περιοχή μέσα στην οποία ένας αστέρα OB μπορεί να διατηρήσει ιονισμένη μέσω της λαμπρότητας του αστέρα και του ρυθμού επανασύνδεσης. Μια τέτοια περιοχή ονομάζεται σφαίρα Stromgren και για μια τυπική θερμοκρασία $10^4 \ K$ και ρυθμό επανασύνδεσης $2 \e{-19} \ m^3 \,s^{-1}$ βρίσκουμε:
\begin{equation}
R_s \simeq 1.7 \, pc \left( \frac{\dot{N}_H}{10^{50} \, s^{-1}} \right)^{1/3} \left( \frac{n_0}{10^9 \, m^{-3}} \right) ^{-2/3}
\end{equation}
όπου $\dot{N}_H$ είναι ο αριθμός των φωτονίων πέρα από το όριο Lyman στη μονάδα του χρόνου και $n_0$ η αριθμητική πυκνότητα των ατόμων υδρογόνου (ανεξαρτήτως κατάστασης).


Στη πραγματικότητα καθώς το σύνορο της περιοχή HII εκκινώντας από τον πρωτοαστέρα με υπερηχητική ταχύτητα, όπως θα δείξουμε παρακάτω, θα ξεπεράσει τελικά την ακτίνα Stromgren λόγω της υψηλότερης θερμοκρασίας (άρα και πίεσης) από το κρύο περιβάλλον του ουδέτερου υδρογόνου. 


Ο χρόνος\footnote{Aν δεχθούμε ότι ο μέσος χρόνος επανασύνδεσης είναι μικρότερος από το χρόνο εξάπλωσης, το οποίο για τις πυκνότητες των περιοχών HII είναι σωστή προσέγγιση} που χρειάζεται για να δημιουργηθεί μια περιοχή HII είναι:
\begin{equation}
t_{expand} \simeq \frac{R_s}{c_{s \, HII}} \simeq 1.7\e{5} \ yr \ \f{\dot{N}_H}{10^{50} \, s^{-1}}{1/3} \f{n_0}{10^9 \, m^{-3}}{-2/3}
\end{equation}
αφού η ταχύτητα του ήχου για τις συνθήκες αυτές, δηλαδή η ταχύτητα εξάπλωσης της περιοχής HII, είναι:
\begin{equation}
c_{s \, HII} = \f{kT_{HII}}{m_{HII}}{-1/2} \simeq 12 \ km\,s^{-1}
\end{equation}


Η ταχύτητα του ήχου για το ουδέτερο αέριο υδρογόνο είναι $c_{s \, HI}\simeq 0.3 \ km \, s^{-1}$, άρα η περιοχή HII δημιουργώντας ένα κρουστικό κύμα καθώς εξαπλώνεται μέσα στο ουδέτερο υδρογόνο. 


Η διαδικασία αυτή είναι πάρα πολύ σημαντική για τη δημιουργία αστέρων. Καθώς το κρουστικό κύμα διασχίζει το μοριακό νέφος το συμπιέζει δημιουργώντας τοπικές συμπυκνώσεις που είναι βαρυτικά ασταθείς με αποτέλεσμα να δημιουργούνται νέοι αστέρες. Στις περιοχές αυτές θα γεννηθούν κάποιοι μαζικοί αστέρες που με τη σειρά τους θα δημιουργήσουν τις δικές τους περιοχές HII και ούτω κάθε εξής.


Το γιγαντιαίο μοριακό νέφος W3 είναι μέρος ενός συμπλέγματος μοριακών νεφών (W3-W4-W5) στον αστερισμό της Κασσιόπης \todo{το χουμε να βρούμε χάρτη του ουρανου?}, σε απόσταση $2 \ kpc$ από τον ήλιο, στη σπείρα του Περσέα του Γαλαξία μας. \todo{σκιτσου γαλαξιακου δισκου με τη τοποθεσία του?}


Η μάζα του εκτιμάται στις $4\e{5} \ M_{\odot}$ κάνοντας το ένα από τα πιο μαζικά μοριακά νέφη στον εξωτερικό Γαλαξία.


Οι περιοχές μεγαλύτερης δραστηριότητας του W3: W3 Main, W3 (OH) και AFGL 333 ανήκουν σε μια μεγαλύτερη δομή πυκνού νέφους (40\% της συνολικής μάζας του W3) υπό την ονομασία HDL (High Density Layer). Αυτή η δομή είναι αποτέλεσμα του κρουστικού κύματος μιας διευρυμένης περιοχής HII (W4 superbubble) που βρίσκεται στα ανατολικά του W3 μέσω του μηχανισμού που περιγράψαμε στη παράγραφο \ref{par:HII regions}. Η περιοχή αυτή τροφοδοτείτε από αστρικούς ανέμους ενός σμήνους OB αστέρων (IC 1805 OB association) που βρίσκονται στη καρδιά του μοριακού νέφους W4.


Πιθανόν αποτέλεσμα της W4 HII είναι η δημιουργία του νεαρού αστρικού σμήνους IC 1795 (με ηλικία $3-5 \ Myr$). To IC 1795 διαθέτει μερικούς αστέρες OB που έχουν δημιουργήσει ένα κέλυφος στο οποίο ανήκουν οι περιοχές W3 Main και W3 (OH).


Στη περιοχή W3 Main έχουν ανιχνευτεί πολλές πυκνές περιοχές HII που αποδίδονται σε νεαρούς αστέρες OB που δεν έχουν διαλύσει ακόμα τα κελύφη σκόνης που τους καλύπτουν. Παρότι φαίνεται ότι η W3 Main διεγέρθηκε από το IC 1795 ή τη W4 HII υπάρχουν ενδείξεις ότι μπορεί να έχει ξεκινήσει τη δημιουργία αστέρων νωρίτερα από το IC 1795. Το οποίο σημαίνει ότι μπορεί η δημιουργία αστέρων να ξεκίνησε αυθόρμητα.


Νότια από το αστρικό σμήνος IC 1795 βρίσκεται η περιοχή W3 (OH) για την οποία υπάρχουν οι ισχυρότερες ενδείξεις ότι αποτελεί προϊόν της πίεσης από το IC 1795 και τη  W4 HII. Στη περιοχή αυτή έχουμε ισχυρές εκπομπές \ce{OH} και \ce{H2O} (masers) από τις οποίες έχει μετρηθεί η απόσταση του μοριακού νέφους μέσω παράλλαξης ($1.95 \pm 0.04 \ kpc$).


Στη νότια-ανατολική γωνία του W3 και κοντινότερα στο W4 βρίσκεται η περιοχή AFGL 333 η οποία παίρνει το όνομα της από τον ομώνυμο αστέρα φασματικού τύπου B0.5 στο εσωτερικό της. 


Εκτός από τις παραπάνω περιοχές όπου πιστεύουμε ότι η διαδικασία δημιουργίας αστέρων είναι πιθανό αποτέλεσμα περιοχών HII, στη νότια-δυτική γωνία του W3 έχουμε τη πιο ισχυρή ένδειξη αυθόρμητης γέννησης αστέρων, τον αστέρα μεγάλης μάζας VES 735 με φασματικό τύπο O8.5 και ηλικία $1-2 \ Myr$ στον οποίο οφείλεται η περιοχή HII KR 140. Το χαρακτηριστικό σφαιρικό κέλυφος της KR 140 είναι διεγερμένη περιοχή όπου δημιουργούνται νέοι αστέρες. 


Βόρεια της KR 140 βρίσκεται η περιοχή KR 140-N ή Trilobite. Η μορφολογία 


\begin{table}
	\caption{Παράμετροι υποπεριοχών του W3 από χαρτογράφηση του Spitzer. YSO: Υoung Stellar Object, συντομογραφία των υποψήφιων πρωτοαστέρων}
	\label{tab:ISM}
	\begin{tabular}{p{2.cm} p{2.5cm} p{1.5cm} p{2cm} p{2.2cm} p{2.5cm}}
		\toprule
		\multirow{2}{*}{Περιοχή} & Μέση απόσταση YSO & Εμβαδόν περιοχής & Μάζα Αερίου & Πυκνότητα Αερίου & Μέση Πυκνότητα YSO \\ 
		&  $(pc)$ & $(pc^2)$ & $(\e{4} \ M_{\odot})$ & $(M_{\odot} \ pc^{-2})$ & $ (pc^{-2}) $ \\
		\midrule
		All-Survey & $0.33 \pm 0.01$ & $1316$ & $6.2\pm0.005$ & $59.44\pm 0.04$ & $1.73\pm 0.001$ \\
	  W3 Main/(OH) & $0.26 \pm 0.01$ & $231.5$ & $1.4\pm0.002$ & $61.31\pm 0.08$ & $3.41\pm 0.005$ \\
	  KR 140 & $0.38 \pm 0.02$ & $853$ & $3.8\pm0.004$ & $45.38\pm 0.04$ & $1.25\pm 0.004$ \\
	  AFGL 333 & $0.34 \pm 0.02$ & $231.5$ & $1.0\pm0.002$ & $53.3\pm 0.1$ & $1.77\pm 0.004$ \\  
		\bottomrule		
	\end{tabular}
\end{table}
